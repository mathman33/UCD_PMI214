\documentclass{article}


\usepackage[margin=0.6in]{geometry}
\usepackage{amssymb, amsmath, amsfonts}
\usepackage{mathtools}
\usepackage{physics}
\usepackage{enumerate}
\usepackage{array}
\newcommand{\Rl}{\mathbb{R}}
\newcommand{\f}[3]{#1\ :\ #2 \rightarrow #3}

\title{PMI 214 Notes}
\author{Sam Fleischer - Speaker: Greg Lanzaro (UC Davis)}
\date{October 20, 2016}

\begin{document}
    \maketitle

    \section{Bluetongue}
    \begin{itemize}
        \item Bluetongue is endemic in North America
        \item From Southern Africa
        \item Vaccine development started in 1902
        \item McKercher (UC Davis) found Bluetongue in UC Davis
        \item generally not contageous
        \item Vector is a midge
        \item Cattle amplify the virus, sheep get sick
        \item segmented genome (very heterogeneic)
        \item policy cost the US billions - the ONLY thing that came out of Africa was diagnostic technology.  To this day, US can't export live animals because of bluetongue, but the paradox is that its already global
        \item 30 species of Culicoides are vectors for bluetongue
        \item climate change predicted bluetongue in spain.. but not france/germany/etc.  But today its in norway and sweden!
        \item NAFTA is based on bluetongue?!!??!
        \item Europe didn't have it until 1998... outbreak in Tunisia, midges flew with a dust storm to Sardinia.  This happened many times in the past, but this time it spread due to climate change.  Caused disease and \emph{social unrest}.
        \item Europe suddenly became receptive to the virus
        \item Oxford modeling group has concluded climate is the main driver
        \item midges are incredibly widely distributed
        \item biological vector, not mechanical (must replicate in the vector)
        \item california distribution caused by human/management/environmental factors - not simply climate
        \item ``overwintering'' of BTV.. what causes it?
        \item virus shows up in the vector in the middle of winter! so it's the insect which is the key to understanding the biology of the virus
        \item not a persistent infection - it is a prolonged viremia
        \item horrific effects on sheep - animals essentially drown in their own fluid
        \item very genetically diverse virus
    \end{itemize}

    \section{African Horse Virus}
    \begin{itemize}
        \item epidemic in the 1960s killed an estimated 300,000 horses
        \item spread all the way to India
        \item In the 1850s, an epidemic caused death in 50\% of all horses in the area
        \item worst disease he's ever seen - discolored to dead in 3 days
        \item Culicoides transmits
    \end{itemize}

\end{document}
















