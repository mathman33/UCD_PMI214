\documentclass{article}


\usepackage[margin=0.6in]{geometry}
\usepackage{amssymb, amsmath, amsfonts}
\usepackage{mathtools}
\usepackage{nicefrac}
\usepackage{physics}
\usepackage{enumerate}
\usepackage{array}
\newcommand{\Rl}{\mathbb{R}}
\newcommand{\f}[3]{#1\ :\ #2 \rightarrow #3}

\title{PMI 214 Notes - Final Review with Greg Lanzaro}
\author{Sam Fleischer}
\date{December 1, 2016}

\begin{document}
    \maketitle

    \begin{itemize}
        \item Lec. 11
        \begin{itemize}
            \item EIR: entomological innoculation rate
            \begin{itemize}
                \item $\text{EIR} = ma\phi\theta$ where
                \begin{itemize}
                     \item $m$ - How many mosquitoes out there? mosq/person
                     \item $a$ - How many bites? bites/day
                     \item $\phi$ - How many bites are on humans (how anthropophilic)? human-bites/all-bites 
                     \item $\theta$ - How many vectors are infected? transmitting-vectors/all-vectors
                \end{itemize}
            \end{itemize}
            \item Vectorial Capacity - number of infective vector bites that would arise from all the mosquitoes that bite a single host on a single day $C = \dfrac{ma^2b}{-\ln p}p^n$
            \begin{itemize}
                \item $a$ - bites/person/day (not the same $a$ as in EIR)
                \item $b$ - vector competence
                \item $n$ - extrinsiv incubation period
                \item $p$ - vector daily survival rate
            \end{itemize}
            \item Basic Reproductive Rate ($R_0$) - avg. \# of future host infections that will arise following introduction of a single infectious host in a susceptible population $R_0 = C/r$ where
            \begin{itemize}
                \item $r$ - host recovery rate 1/infectious-perioud
            \end{itemize}
        \end{itemize}
        \item Lec 12
        \begin{itemize}
            \item Integrated Vector Control (IVM) - think about all the availble tools for the specific vector and situation - ``a rational decision-making process for the optimal use of resources for vector control''
            \item Behavioral Insecticide Resistance - ex: mosquitoes feeding over a range of times, kill all the ones who feed at night, going to select for ones who don't feed at night, bed nets stop being effective.
            \item Source reduction - getting rid of some feature of the environment which the mosquito needs - filling potholes, getting rid of standing water, covering drainage systems, getting rid of carbohydrates, unneeded vegetation, draining swamps
        \end{itemize}
        \item Lec 13
        \begin{itemize}
            \item GMM for vector control
            \item population suppression vs. population replacement
            \begin{itemize}
                \item suppression - reduce fecundity of the population by killing them or getting rid of all the females
                \item replacement - introduce mosquito resistance to malaria through gene drive (turn the vectors into nonvectors)
            \end{itemize}
            \item transgene (in the lab)
            \begin{itemize}
                \item need to have an effector gene (affects the phenotype that kills parasite)
                \item need to have a promoter (turns the gene on at the right time (larval stage, adult stage) in the right place (salivary gland, epidermis lining the stomach, , sex specific, etc))
            \end{itemize}
            \item transgene in nature
            \begin{itemize}
                \item all of the above, and a gene drive (not normal Mendelian inheritance)
            \end{itemize}
            \item How does a gene drive work?
            \begin{itemize}
                \item progeny has one copy of transgene and one copy of wild type.
                \item gene drive cuts the wild type gene and repairs it with the transgene.
                \item heterozygote turns to homozygote
                \item ALL future progeny carries the transgene
            \end{itemize}
        \end{itemize}
        \item Lec 14
        \begin{itemize}
            \item Ecology of Rickettsiaceae
            \item hard and soft tick feeding behavior
            \begin{itemize}
                \item hard ticks feed once per stage
                \item hard ticks have different hosts for each stage
                \item hard ticks feed for a week at a time - they cement themselves to the host, until it gets completely full
                \item soft ticks feed for 20 min at a time
            \end{itemize}
            \item Rickettsiaceae
            \begin{itemize}
                \item bacteria
                \item live inside the cell - not exposed to the host immune system
            \end{itemize}
            \item How do horses get infected with Neorickettsia?
            \begin{itemize}
                \item acquatic insects on wet grass (Carter: ``I think it's... wet food?'')
            \end{itemize}
        \end{itemize}
        \item Lec 15
        \begin{itemize}
            \item Wolbachia
            \begin{itemize}
                \item Wolbachia is very common bacteria in insects - often symbiotic
                \item transmitted maternally (vertical transmission)
                \item also horizontally transmitted (between different species) - mechanism not understood - but good be good for mosquito control, but could cause problems (evolution people say horizontal transmission is between species, not necessarily within species)
            \end{itemize}
            \item Cytoplasmic incompatibility - uninfec/infec female with uninfec/infec male different results.
            \begin{itemize}
                \item infected female with whoever, all infected, normal progeny numbers
                \item infected male with uninfected female, all uninfected, reduced progeny numbers
            \end{itemize}
            \item How are Wolbachia spread?
            \item Bartonian vs.~Fisherian waves
            \begin{itemize}
                \item Fisherian (1937) - pulled - more robust spreading
                \item Bartonian (1979) - threshold - bistable, pushed - easily stopped by barriers to dispersal
            \end{itemize}
        \end{itemize}
        \item Lec 16
        \begin{itemize}
            \item genome evolution in malaria vectors in response to vector control
            \item what is a sibling species?
            \item ``genomic island of speciation'' - in these islands are genes responsible for reproductive isolation
            \item adaptive introgression is one gene moving into another which imparts a fitness advantage
            \item shows up in Anopheles coluzii and An. gambiae
            \begin{itemize}
                \item movement of a good (adaptive) gene from one species to another, or one diverged population to another
                \item what was the gene that moved from coluzii to gambiae?  insecticide resistance.
            \end{itemize}
            \item selective sweep
            \begin{itemize}
                \item if selection is operating on a gene, there is a loss of heterozygosity around that gene
            \end{itemize}
        \end{itemize}
        \item Lec 17
        \begin{itemize}
            \item Leishmaniasis
            \item visceral, cutaneous, mucocotaneous
            \item sand flies
            \item makes the sand flies regurgitate, making it feed more
            \item sand fly saliva induces blood flow, preventing coagulation, and suppresses the immune system
        \end{itemize}
        \item Lec 18
        \begin{itemize}
            \item mechanisms of arbovirus emergence
            \begin{itemize}
                \item 1992-93 vs 1996 outbreaks of VEEV
                \begin{itemize}
                    \item 92-93 - mutation in the virus that changed the pathogenicity to the horses
                    \item 1996 - caused a switch in the vector making it a really good vector
                \end{itemize}
                \item common causes of alphavirus outbreaks?
                \begin{itemize}
                    \item either alter pathongenicity, or
                    \item alters the vector
                \end{itemize}
                \item what determines how frequently these outbreaks occur? environmental conditions
                \item competitive fitness assays?
                \begin{itemize}
                    \item co-infect a host with a mixture and see which one wins in the individual
                    \item if one is over-represented, it is more fit.
                \end{itemize}
            \end{itemize}
        \end{itemize}
    \end{itemize}

\end{document}
















