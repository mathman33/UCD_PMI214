\documentclass{article}


\usepackage[margin=0.6in]{geometry}
\usepackage{amssymb, amsmath, amsfonts}
\usepackage{mathtools}
\usepackage{nicefrac}
\usepackage{physics}
\usepackage{enumerate}
\usepackage{array}
\newcommand{\Rl}{\mathbb{R}}
\newcommand{\f}[3]{#1\ :\ #2 \rightarrow #3}

\title{PMI 214 Notes - Lark Coffey}
\author{Sam Fleischer}
\date{November 29, 2016}

\begin{document}
    \maketitle

    \begin{itemize}
        \item Arboviruses can change their hosts (their ranges, etc)
        \item mechanisms or arboviral emergence
        \begin{itemize}
            \item urban epidemic
            \item enzootic (which can sometimes turn into urban epidemic)
            \item rural epizootic
        \end{itemize}
        \item importance of studyinf alphaviruses
        \begin{itemize}
            \item significant human and veterinary pathogens
            \item affect medically underserved areas of the world
            \item biology relatively poorly studied
            \item lack of effective vaccines
            \item analgesics (painkillers) are only treatment
            \item potential use of bioweapons
        \end{itemize}
        \item Alphavirus case study: VEEV outbreak in 1992-93
        \begin{itemize}
            \item VEEV causes significant disease in humans and equines (can kill a horse in about 4-5 days)
            \item first isolated in the 1940s
            \item This outbreak was in Venezuela
            \item Enzootic: uses cotton rats and spiny rats (horses and humans dead end)
            \item Epizootic: uses horses (humans dead end)
            \item convergent evolution - on 4 different occasions, human and horse infectivity evolved independently
            \item a SINGLE base pair change caused the outbreak.. wow
        \end{itemize}
        \item Alphavirus case study: VEE outbreak in 1993,96
        \begin{itemize}
            \item Hypothesis 1: VEE virus recently introduced
            \item seropositivity increases with age.  This tells us the virus had been there for a while.
            \item Hypothesis 2: VEE emerged via adaptation
            \item some died from fatal encephalitis, but not high enough viremia
            \item a single base pair change caused this outbreak too
        \end{itemize}
        \item Chikungunya, Indian Ocean Islands, 2005-06
        \begin{itemize}
            \item enzootic: aedes $\iff$ non-human cycle
            \item urban: humans have high enough viremia
            \item a single base pair mutation in chik virus affects vector specificity and epidemic potential (mosquito-specific)
        \end{itemize}
        \item What are the mechanisms of global ZIKV emergence including increased severe human disease?
        \begin{itemize}
            \item H1: increased vector transmissibility
            \begin{itemize}
                \item Asian lineage strain has \emph{higher} transmissibility, so H1 is disputed?
            \end{itemize}
            \item H2: increased human virulence
            \begin{itemize}
                \item no evidence for this
            \end{itemize}
            \item H3: no change in mosquitoes or humans; emergence results from a stochastic effect (more people infected, well-characterized outbreaks)
            \begin{itemize}
                \item Lark's favorite
                \item possible nobody was looking?
            \end{itemize}
            \item H4: increased human susceptibility
        \end{itemize}
    \end{itemize}

\end{document}
















