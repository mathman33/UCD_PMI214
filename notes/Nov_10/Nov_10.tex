\documentclass{article}


\usepackage[margin=0.6in]{geometry}
\usepackage{amssymb, amsmath, amsfonts}
\usepackage{mathtools}
\usepackage{physics}
\usepackage{enumerate}
\usepackage{array}
\newcommand{\Rl}{\mathbb{R}}
\newcommand{\f}[3]{#1\ :\ #2 \rightarrow #3}

\title{PMI 214 Notes - Janet Foley}
\author{Sam Fleischer}
\date{November 10, 2016}

\begin{document}
    \maketitle

    \section*{Tick Talk}

    \begin{itemize}
        \item tick borne pathogens
        \item ticks are arthropods - jointed legs
        \item ticks are ``obligated'' parasites - must eat blood
        \item no ticks can fly
        \item ticks have 8 legs like spiders
        \item mouth parts tucked underneath
        \item soft ticks feed once for 20 minutes every week
        \item hard ticks feed once per stage.  It must transform in to a new stage before it transmits the disease.  But the tick only needs to feed once in a year!
        \item some diseases in some ticks are trans-ovarial - adult tick gets infection and passes it on to child
        \item Wolbachia are ubiquitous parasites
        \item Wolbachia can either
        \begin{itemize}
            \item really really damage their host, or
            \item actually help their host.  Wolbachia is a parasite in heartworm, which is a parasite in dogs..
            \begin{itemize}
                \item When you kill the heartworm in the dog, you may release wolbachia.  So first vets treat the worm, then they kill the worm.
            \end{itemize}
        \end{itemize}
        \item Potomac horse fever - caused by neorickettsia
        \item Salmon poisoning - same as above, but affects dogs..
        \begin{itemize}
            \item bacteria $\rightarrow$ fluke $\rightarrow$ salmon $\rightarrow$ dog
        \end{itemize}
        \item Tick prevalence is dependent on anthropogenic change (climate change)
        \item rocky mountain spotted fever (rickettsia rickettsii)
    \end{itemize}

\end{document}
















