\documentclass{article}


\usepackage[margin=0.6in]{geometry}
\usepackage{amssymb, amsmath, amsfonts}
\usepackage{mathtools}
\usepackage{physics}
\usepackage{enumerate}
\usepackage{array}
\newcommand{\Rl}{\mathbb{R}}
\newcommand{\f}[3]{#1\ :\ #2 \rightarrow #3}

\title{PMI 214 Notes - ``Epidemiology of Canine Heartworm (\emph{Dirofilaria immitis})''}
\author{Sam Fleischer - Speaker: Ben Sacks (UC Davis - Population Health and Reproduction Department)}
\date{September 29, 2016}

\begin{document}
    \maketitle

    \section{Key concepts}
        \begin{itemize}
            \item Helminth parasites: endemic, not epidemic
            \begin{itemize}
                \item Once they're within a host population, there may be fluctuations, but no huge eruptions
            \end{itemize}
            \item Environmental factors determine potential for transmission
            \begin{itemize}
                \item climate
                \item host presence
                \item vector habitat
            \end{itemize}
            \item Changes in the environment change that potential
        \end{itemize}
    \section{Why do we care about canine heartworm?}
        \begin{itemize}
            \item dogs
            \begin{itemize}
                \item mild: coughing
                \item moderate: difficulty breathing
                \item serious: pulmonary hypertension, caval syndrome, RSCHF
            \end{itemize}
            \item cats
            \begin{itemize}
                \item rare hosts but poor prognosis
            \end{itemize}
        \end{itemize}
    \section{lifecycle}
        \begin{itemize}
            \item lodged in pulmonary artery
            \item give birth to microfilaria (not eggs) which circulte in the bloodstream (300 um)
            \item ingested by mosquito (if lucky)
            \item mults into invective larvae
            \item mosquito transmits to definitive host
            \item mults some more into a reproductive adult
        \end{itemize}

    \section{At $26^\circ$}
        \begin{itemize}
            \item 14 - 18 days in the vector
            \item they chill in the needle of the mosquito and sloppily fall out when mosquito bites a canid host
        \end{itemize}

    \section{In the canid}
        \begin{itemize}
            \item 3-12 days to L4
            \item larger and larger
            \item $\sim$180 days to sexual maturity
        \end{itemize}

    \section{Efficient vectors in CA}
        \begin{itemize}
            \item 
        \end{itemize}

    \section{Where do we find heartworm in CA?}
        \begin{itemize}
            \item evidence that western tree hole mosquito is the main vector
        \end{itemize}

    \section{Coyotes are excellent sentinels}
        \begin{itemize}
            \item Widespread
            \item Relatively stationary (as opposed to dogs, which travel with humans)
            \item prophylaxis
            \item always outdoors
            \item high susceptibility
        \end{itemize}

    \section{two studies}
        \begin{itemize}
            \item high-incidence zone
            \item building and testing a predictive spacial risk model
        \end{itemize}

    \section{What variavles influence transmission?}
        \begin{itemize}
            \item definitive host presense/abundance
            \item vector presence/abundance
            \item parasite presence
            \item temperature
        \end{itemize}

    \section{Western treehole mosquito}
        \begin{itemize}
            \item feed in treeholes where water can gather
            \item april/may: adults emerge from pupae in tree holes - photoperiod triggered
            \item females only take 1 blood meal per clutch of eggs
            \item most females die before taking a second blood meal, approx. 2 weeks later
            \item some females may have 4 or 5 clutches (and blood meals) before dying at the end of summer
            \item temperature
            \begin{itemize}
                \item heartworm larvae requires $>14^\circ$ to develop
            \end{itemize}
        \end{itemize}

    \section{Questions}
        \begin{itemize}
            \item What limits transmission season?
            \item how long/variable is the season?
            \item does transmission intensity vary annually?
        \end{itemize}

    \section{How to date a transmission event?}
        \begin{itemize}
            \item determine heartworm age
            \begin{itemize}
                \item How?  use length as a proxy
            \end{itemize}
            \item subtract time from sampling date
            \item arrive at date when the infective L3 entered canine host
            \begin{itemize}
                \item Most transmission occurs in the span of 1 month
                \item transmission starts about 1.5 months after temperature enters desirable zone
            \end{itemize}
        \end{itemize}

    \section{Transmission, rain, and vector abundance correlated}
        \begin{itemize}
            \item Vector abundance correlates with transmission and also rainfull
        \end{itemize}

    \section{Conclusions}
        \begin{itemize}
            \item warming determines onset of transmission
            \item waning vector abundance ends transmission
            \item transmission occurs within a month
            \item precipitation affects vector abundance, which affects transmission
        \end{itemize}

    \section{How do we make broad spatial predictions?}
        \begin{itemize}
            \item (Epi) map parasite occurence and describe and loor for commonalities, or
            \item (Eco) map parasite prevalence and quantify relationships with variables
        \end{itemize}

    \section{CA Risk map}
        \begin{itemize}
            \item heterogeneous distribution
        \end{itemize}

    \section{Modeling approach}
        \begin{itemize}
            \item Logistic regression
            \item dependent var: seropositivity (0 or 1)
            \item independent vars:
            \begin{itemize}
                \item temp
                \item precipitation
                \item distance from prime habitat
            \end{itemize}
        \end{itemize}

    \section{Final model}
        \begin{itemize}
            \item HDU (+)
            \item Precipitation (+)
            \item distace from dense oak woodland
            \item HDU*Precipication
        \end{itemize}

    \section{External Validation Sites}
        \begin{itemize}
            \item pro of noting a ton of data - extrenal validation
        \end{itemize}

    \section{Conclusions}
        \begin{itemize}
            \item model confirms varaibles on broad spacial scale
            \item Locally important vectors, risk factos also matter but are not incorporated in the model
            \item bottom line: much easier to predict ``risky'' areas instead of ``safe'' areas
        \end{itemize}

\end{document}
















