\documentclass{article}


\usepackage[margin=0.6in]{geometry}
\usepackage{amssymb, amsmath, amsfonts}
\usepackage{mathtools}
\usepackage{physics}
\usepackage{enumerate}
\usepackage{array}
\newcommand{\Rl}{\mathbb{R}}
\newcommand{\f}[3]{#1\ :\ #2 \rightarrow #3}

\title{PMI 214 Notes}
\author{Sam Fleischer - Speaker: Lark L.~Coffey (UC Davis)}
\date{October 11, 2016}

\begin{document}
    \maketitle

    \begin{itemize}
        \item First found in humans in Serologic tests (pre-genomic era)
        \item initially isolated in forests in Africa
        \item various monkeys were involved in the spread of Zika
        \item zika is a flavivirus
        \item sylvatic cycles in forests lead to emergence in urban settings
        \item Humans serve as good amplifiers of Zika
        \item Once Zika gets into an urban setting:
        \begin{itemize}
            \item Vertical transmission
            \item Sexually transmitted F-F, F-M, M-M
            \item Contact Transmission
            \item Main form of transmission is via mosquitos
        \end{itemize}
        \item mojority of cases are asymptomatic
        \item common symptoms:
        \begin{itemize}
            \item red eyes
            \item fever
            \item rash
            \item joint pain
        \end{itemize}
        \item Microcephaly in Brazil
        \item Zika has exposed gaps in medical tech in impoverished areas
        \item Heads are two standard deviations below the mean
        \item two major lineages: African and Asian
        
    \end{itemize}

\end{document}
















