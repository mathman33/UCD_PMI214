\documentclass{article}


\usepackage[margin=0.6in]{geometry}
\usepackage{amssymb, amsmath, amsfonts}
\usepackage{mathtools}
\usepackage{physics}
\usepackage{enumerate}
\usepackage{array}
\newcommand{\Rl}{\mathbb{R}}
\newcommand{\f}[3]{#1\ :\ #2 \rightarrow #3}

\title{PMI 214 Notes - ``Surveillance and control of vectors in California''}
\author{Sam Fleischer - Speaker: Paula Macedo (Sacramento-Yolo Mosquito and Vector Control District)}
\date{September 27, 2016}

\begin{document}
    \maketitle

    \section{Key concepts}
        \begin{itemize}
            \item What makes CA unique
            \item funding..
            \item others
        \end{itemize}

    \section{CDPH}
    \begin{itemize}
        \item Rodent-Borne - Hantavirus
        \item Flea - Typhus and Plague
        \item Tick - Lyme among others
        \item Mosquito - 
        \item Cooperative agreement with local agencies

        \item Sac-Yolo mostly works with ticks and mosquitos
    \end{itemize}

    \section{Cooperative Agreement}
    \begin{itemize}
        \item Calibration of equipment and record keeping
        \item reports about pesticides and adverse effecs
        \item technician certification
        \begin{itemize}
            \item many requirements
        \end{itemize}
    \end{itemize}

    \section{Sac-Yolo Mosquito and Vector Control District}
    \begin{itemize}
        \item one of the three biggest in CA along with LA and OC
        \item funding?
        \begin{itemize}
            \item bi-county - funded from property taxes in both counties
            \item other agencies (ie SD) funded from public health dept
            \item some agencies are forced to ask the public to charge them more if there is a big outbreak
        \end{itemize}
        \item governed by board of trustees
        \begin{itemize}
            \item make decisions for what's best for cities and counties
            \item 1915 - Mosquito Abatement Act
            \item operate under CA health and safety code
            \begin{itemize}
                \item they have authority to enter a premises if there is a reason to believe there is a health concern
            \end{itemize}
        \end{itemize}
    \end{itemize}

    \section{Integrated Mosquito Management}
    \begin{itemize}
        \item Surveillance
        \item Control
        \begin{itemize}
            \item Physical
            \begin{itemize}
                \item Easier to empty a pool than to continue to implement pesticides.  Unattended pools could produces 100,000s of mosquitos
            \end{itemize}
            \item Biological
            \item Checmical
            \begin{itemize}
                \item Larvicide - killing mosquito in water
                \item Adulticide
            \end{itemize}
        \end{itemize}
        \item Education
        \begin{itemize}
            \item \$500,000/yr
        \end{itemize}
    \end{itemize}

    \section{Surveillance}
    \begin{itemize}
        \item Traps
        \begin{itemize}
            \item Light Trap - attracts many other attractor - not effective in urban areas, bc light competition
            \item Magnet Trap - requires propane - gets stolen ALOT - better if your neighbor has it, not you - no longer on the market
            \item Gravid Trap - stinky - alfalfa yeast water... attacting females ready to lay eggs
            \item Ovitrap
            \item EVS trap (Encephalitic)
            \item BG-Sentinel trap - stolen alot - white elephant - very obvious... cost \$200
            \item AGO (autocidal gravid ovitrap)
            \item Resting trap - after blood feeding..
            \item Aspirator
            \item Field samples
            \item 200-300 traps/wk in Sac-Yolo
        \end{itemize}
        \item Encephalitis Virus Surveillance
        \begin{itemize}
            \item Dead bird collections
            \begin{itemize}
                \item Most successful indicator
                \item they take birds even if they died years ago
                \item West Nile is very stable
            \end{itemize}
            \item Sentinel chickens
            \begin{itemize}
                \item worthless
                \item usually viruses are found 8 weeks after they show up in a dead bird
                \item more useful in a remote location with few humans
            \end{itemize}
            \item EVS
            \begin{itemize}
                \item Co2 dry ice and gravitrap
                \item test live mosquitos the next day
            \end{itemize}
            \item Test for West Nile, Western and St. Louis Equine Encephalitis
        \end{itemize}
    \end{itemize}

    \section{West Nile Virus}
    \begin{itemize}
        \item 2 people died this year and this will grow...
        \item a study found that infection risk is 3-20 \%.  From those, 20\% get ill.  From those, only .7\% show encephalitis.  From those, 4-18\% die.  low mortality.
        \item hospitalized patients:
        \begin{itemize}
            \item 37\% recover fully
            \item neurological deficits are long-held
            \item 8 months later, still fatigue, myalgias, headaches, persistent cognitive deficits
        \end{itemize}
    \end{itemize}

    \section{Testing for other pathogens}
    \begin{itemize}
        \item vectors
        \begin{itemize}
            \item mosquitos
            \begin{itemize}
                \item Malaria
                \item dengue
                \item chikungunya
                \item zika
                \begin{itemize}
                    \item sexual transission is important
                    \item some people in Sac-Yolo have it from travelling
                \end{itemize}
                \item This dept checks the residence of victim for mosquitos
            \end{itemize}
            \item ticks and others
            \begin{itemize}
                \item lyme
                \item tularemia
            \end{itemize}
        \end{itemize}
        \item Some mosquitos lay eggs on the edge of plants so that when it floods the eggs can hatch
        \item mosquitos are very easily introduced in any area
        \item mosquitos that transmit some diseases (zika) most likely eat bird blood - they would have to bite a human twice in order to transmit in Sac-Yolo
        \item Tick Surveillance
        \begin{itemize}
            \item flags used to colect ticks
            \item ticks chill at the top of a plant and put their legs up, hoping to snag a dog or your leg.. lyme disease - big problem in CA, not just East Coast
            \item if a tick bites a lizard, lyme is cleared
            \item ticks are by cache creek and by the american river
            \begin{itemize}
                \item Tick nymphs chill on logs.. as many 20\% carry
            \end{itemize}
        \end{itemize}
        \item Yellowjacket Surveillance
        \begin{itemize}
            \item No disease transmission, but a nuisance
            \item chemical lure traps
            \item queen trapping
        \end{itemize}
        \item Africanized Honeybee (killer)
        \begin{itemize}
            \item From Brazil
            \item Swarm trap
            \item chemical lure
            \item story goes the researcher quarentined the bee, but it escaped and eventually got all the way to USA - mostly in So Cal - temperature-limited factor?
        \end{itemize}
        \item Dog Heartworm Surveillance
        \begin{itemize}
            \item dept works with vets because medication is expensive and certain vectors are not in certain places
        \end{itemize}
        \item Pesticide resistance management
        \begin{itemize}
            \item Test for resistance to pesticides.. constantly
        \end{itemize}
    \end{itemize}

    \section{Control}
    \begin{itemize}
        \item kiddy pools are a problem
        \item emptied pools are a problem when it rains
        \item collection basins..
        \item clogged gutters..
        \item in so cal they poke holes and/or take stuff from peoples backyards!
        \item Rice fields - huge mosquito producer
        \item Davis and Woodland are surrounded by rice
        \item Winters is a big problem since it is surrounded by mountains so planes can't effectively drop pesticides
        \item Gambusia
        \begin{itemize}
            \item treat fish with antibiotics
        \end{itemize}
    \end{itemize}

    \section{Education}
    \begin{itemize}
        \item Anyone who will share the message is welcome..
        \item Largest education budget - \$500,000
        \item Ops budget - \$150,000
        \item Do your part
        \begin{itemize}
            \item Drain standing water
            \item Avoid being outdoors at dawn and dusk
            \item dress appropriately
            \item defend yourself
            \item door/window screens
            \item FightTheBite.net
            \item 800-429-1022
        \end{itemize}
    \end{itemize}

    \section{Key concepts}
    \begin{itemize}
        \item CA is unique
        \begin{itemize}
            \item the BEST in terms of mosquito control..
            \item very organized
        \end{itemize}
        \item funded through tax dollars, but operator independently
    \end{itemize}

    \section{Contact}
    \begin{itemize}
        \item pmacedo@fightthebite.net
        \item 916-405-2066
    \end{itemize}

    \section{Use of mathematical models? Stats?}
    \begin{itemize}
        \item Vector index
        \item infection rates
        \item research budget - Dr.~Barker (giving a talk in this class)
    \end{itemize}
        


\end{document}