\documentclass{article}


\usepackage[margin=0.6in]{geometry}
\usepackage{amssymb, amsmath, amsfonts}
\usepackage{mathtools}
\usepackage{nicefrac}
\usepackage{physics}
\usepackage{enumerate}
\usepackage{array}
\newcommand{\Rl}{\mathbb{R}}
\newcommand{\f}[3]{#1\ :\ #2 \rightarrow #3}

\title{PMI 214 Notes - Greg Lanzaro}
\author{Sam Fleischer}
\date{November 22, 2016}

\begin{document}
    \maketitle

    \begin{itemize}
        \item Leishmania has flagella
        \item vectored by sand flies in the family Psychodidae
        \item a few Leishmania species cause visceral leishmaniasis
        \item lifecycle
        \begin{itemize}
            \item infected sand fly
            \item attachment in macrophage in mammal (parasites delivered into the skin, not the blood vessels)
            \item macrophages get into bloodstream (maybe to the liver in the case of visceral leishmaniasis)
            \item change into amastigote, undergo sexual reproduction (survive in a parasitophagous vacuoule)
            \item infected cells rupture and new parasites invade again, or,
            \item into sandfly via bite
            \item infects the fly
        \end{itemize}
        \item 12 million people in 88 countries known infected - 350 million at risk
        \item 2 million (0.5 million visceral) new cases new cases per year
        \item 20 species of Leishmania parasites which cause disease in man
        \item spectrum of clinical disease
        \begin{itemize}
            \item cutaneous
            \item mucu-cutaneous
            \item visceral
        \end{itemize}
        \item visceral (Kala-Azar)
        \begin{itemize}
            \item up to 100\% fatal in some areas
            \item mainly Ethiopia and other African countries, and also cases in Brazil (some from Mexico down to Argentina)
            \item symptoms
            \begin{itemize}
                \item bouts of fever
                \item enlargement of liver and spleen
                \item wasting and weakness
                \item darkening of the skin (kala azar means black fever)
                \item loss of certain immune responses
            \end{itemize}
            \item cutaneous
            \begin{itemize}
                \item self-limiting
                \item systemic signs are absent - it's a local site
                \item legions on skin start out small and could get up to 2cm in diameter.
                \item untreated sores can leave massive scars
            \end{itemize}
            \item muco-cutaneous
            \begin{itemize}
                \item can completely destroy the face
                \item currently much less common
            \end{itemize}
            \item zoonotic disease
            \item humans are (mostly) dead-end hosts
            \item but opposoms, burrowing rodents, wild canids, domestic dogs can infect sandflies
        \end{itemize}
        \item diagnosis
        \begin{itemize}
            \item microscopic examination of lesions
            \item isolation of parasites in culture
            \item most simple: detection of antibodies to the parasites
        \end{itemize}
        \item treatment
        \begin{itemize}
            \item cutaneous - most heal without treatment
            \item other forms are much more difficult to treat - requires long course of treatment with toxic drugs (essentially chemo - many people die from the treatment)
            \item drug resistance is starting to become a problem
        \end{itemize}
        \item DON'T GET THIS DISEASE!  No prophylactics!  Use insect repellents and keep your fingers crossed
        \item prevention
        \begin{itemize}
            \item no vaccine or prophylactic
            \item avoid outdoor activities
            \item protective clothing, repellents
            \item insecticides, bed nets
        \end{itemize}
        \item disease in the sandfly
        \begin{itemize}
            \item most species (400/500) are in the Americas
            \item only 70 are known to carry the disease
            \item sandflies can lay eggs in moist areas.. don't need water
            \item they don't fly very well - mostly flight-assisted hopping
            \item mating biology is complicated
            \item only females bite
            \item like malaria, insect part of lifecycle is most complicated - many stages - metacyclic stage is the one which infects hosts
            \item 0-4 days post-infection
            \begin{itemize}
                \item parasites undergo transformation into flagellated form
                \item attach to lining by the tip of the flagella while blood meal is ingested and excreted (must happen otherwise fly excretes them)
            \end{itemize}
            \item 7-15 days
            \begin{itemize}
                \item replicate while attached to the lining
                \item move toward the mouthparts in the digestive system
                \item move up against a valve which opens and closes to prevent food from being regurgitated
                \item parasites destroy the valve
                \begin{itemize}
                    \item causes the fly to regurgitate while feeding
                    \item these don't infect salivary glands.  rather, they are regurgitated into the wound via destruction of valve
                    \item causes sandflies to bite more often since it's harder for them to feed
                    \item the Plague is similar, but in flease
                \end{itemize}
            \end{itemize}
        \end{itemize}
        \item saliva of blood-feeding arthropods
        \begin{itemize}
            \item blood feeding has evolved at least 16 different independent times
            \item insects have evoled a series of compounds which turn coagulation and other systems off - allows insects to feed on blood without the blood clotting
            \begin{itemize}
                \item 16 different lineages have evoled certain forms of this behavior - studies of bloodfeeders' spit is interesting from a pharmacological perspective
            \end{itemize}
            \item saliva also suppresses immune response
            \begin{itemize}
                \item suppress production of cytokynes
                \item sets up perfect environment for parasites since immune response is already diminished
            \end{itemize}
        \end{itemize}
        \item erythma caused by bite of uninfected Lu.~longipalpis
        \item vasodilation
        \begin{itemize}
            \item local dilation of blood vessels
            \item caused by maxadilan - most potent dilator - 7$\times$ more potent than nitroglycerin (which is standard in the medical community)
        \end{itemize}
        \item Allowing sandflies to feed on mice which were vaccinated against maxadilan reduces their bloodfeeding, so they less eggs, so there is a fitness cost
        \item Sand fly saliva - maxadilan - may be the key to a vaccine
    \end{itemize}

    \section{Key points}
    \begin{itemize}
        \item caused by trypanosomes
        \item zoonotic
        \item spectrum of clinical manifestations
        \item bloodfeeding evolved independently
        \item and more.. see slides
    \end{itemize}

\end{document}
















