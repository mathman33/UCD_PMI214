\documentclass{article}


\usepackage[margin=0.6in]{geometry}
\usepackage{amssymb, amsmath, amsfonts}
\usepackage{mathtools}
\usepackage{physics}
\usepackage{enumerate}
\usepackage{array}
\newcommand{\Rl}{\mathbb{R}}
\newcommand{\f}[3]{#1\ :\ #2 \rightarrow #3}

\title{PMI 214 Notes}
\author{Sam Fleischer}
\date{September 22, 2016}

\begin{document}
    \maketitle

    \section*{Course Overview}
        \subsection*{Ecology, Epidemiology, and Control of Vector-Borne Diseases}
            \begin{itemize}
                \item Sep 27 - Mosquito Control in CA
                \item Sep 29 - Dirofiliaria
                \item Oct 4 - West Nile virus Epidemiology and Ecology
                \item Oct 6 - Impact of Environmental Change on Vector-Borne Desease
                \item Oct 11 - Zika Virus
                \item Oct 13 - Dengue Virus
                \item Oct 18 - Malaria
                \item Oct 20 - African Horse Sickness \& Blue Tongue Viruses
                \item Oct 25 - Midterm Review
                \item Oct 27 - MIDTERM (50\%)
            \end{itemize}
        \subsection*{Surveillance \& Control of Vector-Borne Diseases}
            \begin{itemize}
                \item Nov 1 - Modeling \& surveillance of vector-borne diseases
                \item Nov 3 - Integrated VEctor Control
                \item Nov 8 - Genetically modified mosquitoes for malaria control
                \item Nov 10 - Genome evolution of malaria vectors in response to vector control
                \item Nov 15 - Dengue control by Introducing Wolbachia in A. aegypti populations (Turelli !!!)
                \item Nov 17 - Ecology of Rickettsiaceae
                \item Nov 22 - Leishmaniases
                \item Nov 24 - THANKSGIVING
                \item Dec 3 - Mechanisms of Arbovirus
                \item Dec 6 - FINAL EXAM (50\%) (not cumulative)
            \end{itemize}
    \section*{Lecture 1}
        \subsection*{Objectives}
            \begin{itemize}
                \item Factors affecting efficiency of transmission
                \item parasite acquisition
                \item modes of replication and transmission of different parasites
            \end{itemize}
        \subsection*{Factors Affecting Efficienct of Transmission}
            \begin{itemize}
                \item ``Nidus of Transmission'' - where the host, vector, and parasite come together in a permissive environment
                \item Transmission
                \begin{itemize}
                    \item Mechanical Transmission (typically bacterial pathogens)
                    \begin{itemize}
                        \item Does not involve biological association between pathogen and vector
                        \item typically mouth parts of insects like houseflies
                        \begin{itemize}
                            \item flies eat fecal matter, land on breakfast cereal, and transfer enough to get the kid sick
                        \end{itemize}
                        \item vector serves only in a physical manner
                        \item Maybe this includes copulative transmission?
                    \end{itemize}
                    \item Biological Transmission
                    \begin{itemize}
                        \item ingested parasite either develops and/or reproduces within the arthopod
                        \item example is Malaria - parasite undergoes changes inside the mosquito
                        \begin{itemize}
                            \item Changes from infective to the mosquito into infective to the vertebrate host
                            \item Infected female does not transmit to offspring
                        \end{itemize}
                        \item vector picks up a low titre of particles, viruses reproduce inside the mosquito into numbers so that it infects the salivary glands inside the mosquito
                        \item most successful form of transmission
                        \item trophic transmission (S. Solidus) can be considered a subset of biological transmission (but not everything in biology fits into boxes)
                    \end{itemize}
                    \item Horizontal Transmission (includes mechanical and biological)
                    \begin{itemize}
                        \item transmission between hosts
                        \item dengue: mosquito $\rightarrow$ human $\rightarrow$ mosquito $\rightarrow$ human $\rightarrow \dots$
                    \end{itemize}
                    \item Vertical Transmission
                    \begin{itemize}
                        \item pretty common, but not east for parasite to cross from female to egg
                        \item female ticks pass lyme disease to their offspring
                    \end{itemize}
                \end{itemize}
                \item Incubation
                \begin{itemize}
                    \item Intrinsic Incubation Period
                    \begin{itemize}
                        \item Typical period within vertebrate host between infection and onset of disease
                        \item Very important from and epidemiological (and even legal) standpoint
                        \item Recredescence
                        \begin{itemize}
                            \item parasite becomes sequestered
                            \item if the immune system gets compromise, you get old or sick, have a transplant, then the parasite emerges
                            \item Record is 75 years.. the Greek woman who moved to Minnesota had Malaria
                        \end{itemize}
                    \end{itemize}
                    \item Extrinsic Incubation Period
                    \begin{itemize}
                        \item Period within vector between infection and transmission
                        \item Mechanical transmission: EIP$=0$.
                        \item Biological transmission: EIP$\neq0$.. must infect the salivary glands
                    \end{itemize}
                \end{itemize}
                \item Blood Feeding
                \begin{itemize}
                    \item Blood feeding has evolved independently at least 21 times in arthropods
                    \item Since they've evolved independently, they have created different methods of blood meal acquisition
                    \item Types of mouthparts
                    \begin{itemize}
                        \item No penetration (houseflies) - sponging mouthparts, can only suck up liquid, can't create a surface wound.
                        \item Creates surface wound
                        \begin{itemize}
                            \item rasping, chewing, sponging mouthparts
                            \item heavily armored mouthparts
                            \item blade-like parts create the wound, then use sponge-like parts to suck up blood.
                        \end{itemize}
                        \item Penetrate epidermis to find blood vessels
                        \begin{itemize}
                            \item Mouthparts are seringe-like and enter a blood vessel
                            \item Bed bugs, ``kissing'' bugs, and true bugs have evolved anesthetic so the victim doesn't feel the bite
                        \end{itemize}
                    \end{itemize}
                \end{itemize}
            \end{itemize}
\end{document}