\documentclass{article}


\usepackage[margin=0.6in]{geometry}
\usepackage{amssymb, amsmath, amsfonts}
\usepackage{mathtools}
\usepackage{physics}
\usepackage{enumerate}
\usepackage{array}
\newcommand{\Rl}{\mathbb{R}}
\newcommand{\f}[3]{#1\ :\ #2 \rightarrow #3}

\title{PMI 214 Notes}
\author{Sam Fleischer}
\date{October 25, 2016}

\begin{document}
    \maketitle

    \section{Review}
    \begin{itemize}
        \item What are the components of vector-borne disease?
        \item extrinsic vs intrinsic incubation?
        \item of the pathogens, know who vectors what and generally where they occur, also, who is the vertebrate reservoir.  Do Zika and Chikun have avian reservoirs?
        \item both invertebrate and vertebrate a host.  a reservoir is conventially the vertebrate and a vector is conventially the invertebrate.  both are hosts.
        \item learn about heartworm
        \item infections start earlier when temperature warms earlier.
        \item transmission of heartworm lasts one month because of intrinsic biology of the worm
        \item HDU is a temperature index
        \item surprise - the worm appears where the mosquito lives
        \item sampling bias tells us its easier to predict risky areas than to predict safe areas
        \item 
        \item West Nile and Zika - many people are asymptomatic - really good to check blood banks for disease
        \item humans are dead end hosts for West Nile (not for Zika and Chikun) because of low viremia (lab studies)
        \item every horse in the US gets the West Nile vaccine
        \item 
        \item human demography affects vectorborne diseases
        \item increasing temperature decreases time till viremic in disease (main case for why climate change will increase human diseases)
        \item 
        \item the challenge of having two flaviviruses (DENV and ZIKV) in the same area
        \item 
        \item bluetongue is extremely easily transmitted by midges - via dust storms
    \end{itemize}

\end{document}
















