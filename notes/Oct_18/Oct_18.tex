\documentclass{article}


\usepackage[margin=0.6in]{geometry}
\usepackage{amssymb, amsmath, amsfonts}
\usepackage{mathtools}
\usepackage{physics}
\usepackage{enumerate}
\usepackage{array}
\newcommand{\Rl}{\mathbb{R}}
\newcommand{\f}[3]{#1\ :\ #2 \rightarrow #3}

\title{PMI 214 Notes}
\author{Sam Fleischer - Speaker: Greg Lanzaro (UC Davis)}
\date{October 18, 2016}

\begin{document}
    \maketitle

    \begin{itemize}
        \item Malaria no longer in US because of policies and culture in the 1950s (window screens, new deal stuff, A/C and TV)
        \item Malaria is mostly ($>90\%$) a problem in sub-Saharan Africa.  In the New World, mostly in Brazil
        \item Malaria is mainly transmitted by Anopheles mosquitoes - many different types - many different biologies
        \item Makes Malaria control through mosquito control very difficult
        \item $>200,000,000$ clinical cases per year
        \item $>600,000$ deaths per year ($>90\%$ in Africa)
        \item increasing problem
        \begin{itemize}
            \item evolution of resistance of Malaria parasites
            \item mosquito resistance to insecticides
            \item no economic incentive to produce treatments
        \end{itemize}
        \item 4 human Malaria parasites
        \begin{itemize}
            \item Plasmodium $\underline{\hspace{1cm}}$
        \end{itemize}
        \item antibodies do NOT provide immunity like they do with Dengue
        \item origins of Malaria traced back to the age of dinosaurs (someone found a bloodfeeding fly in amber just this year)
        \item every group of vertebrates has its own set of malaria parasites
        \item Humans have had Malaria since the beginning of homo-sapiens
        \item Hippocrates wrote about symptoms of Malaria
        \item comtemporary knowledge began in 1880 when Laveran found malaria in the parasite - Ross in 1897 found it was transmitted by mosquitoes
        \item three stages of the malaria parasite life cycle
        \begin{itemize}
            \item liver stage (site of initial infection)
            \item blood stage (produces pathology)
            \item mosquito stage
        \end{itemize}
        \item Liver stage
        \begin{itemize}
            \item invade hepatocytes in the liver
            \item asexual reproduction
            \item highly virulent
            \item no overt pathology
            \item 6-15 days
            \item sometimes Malaria parasites hibernate in the liver (record in 75 years) (just happens with \emph{P. vivax} and \emph{P. ovale})
        \end{itemize}
        \item Blood stage
        \begin{itemize}
            \item enter the red blood cell
            \item ingest contents
            \item replicate
            \item rupture and releases merozoite
            \item develop anemia
            \item infected cells stick to vessel walls - then bursts
            \item symptoms
            \begin{itemize}
                \item headache
                \item fever
                \item fatigue
                \item pain
                \item very general symptoms
            \end{itemize}
            \item general symptoms can lead to deadly situations within 24 hours
            \item clincial features:
            \begin{itemize}
                \item periodic episodes of fever (red blood cells all rupture around the same time - very regular - can be diagnostic)
            \end{itemize}
            \item infected cells produce adhesive knobs - sticks to the lining of the blood vessels - can clump and cause clogs
            \item infected cells release toxins (ex: nitric oxide, which is extremely damaging to brain cells) $\leftarrow$ this is the main killer of humans
        \end{itemize}
        \item Mosquito Stages
        \begin{itemize}
            \item Malaria reproduces sexually in the mosquito
            \item Mosquito will simply ingest human malaria parasites
            \item Parasites become gametocytes, which, within \emph{hours} after ingestion by mosquito turn in to gametes and form zygotes
            \item The mosquito is, in fact, infected
            \item This cycle takes 2 weeks
            \item Macrogamete(femele) are inseminated by microgametes(male)
            \item ookinete is the only diploid stage - exits the stomach by penetrating stomach cells and produces cists which release parasites, which migrate and penetrate into the salivary gland
        \end{itemize}
        \item Malaria puts a huge economic burden on developing countries - sick people can't come into work
        \item Malaria has driven human evolution in some cases
        \begin{itemize}
            \item Homozygotes for anemia have big problems
        \end{itemize}
        \item Malaria pills damage liver if taken more than 6 months to a year.. but in the short term, can prevent Malaria spread in the human body
        \item Management
        \begin{itemize}
            \item insecticide-treated bed nets
            \item indoor residual spraying (DDT indoors... yuck... stays for more than a year)
            \begin{itemize}
                \item environmental impacts are fairly minimal if this is restricted to indoors?
                \item relatively nontoxic to humans
                \item people accept is
                \item water solubale (don't have to mix it with oil)
                \item kills cockroaches, ants, mosquitoes, and other stuff
            \end{itemize}
            \item Bed nets and drugs work well (Gates foundation, George Bush was good for Malaria (President's Malaria Initiative)) - have saved hundreds of thousands of lives
            \item Genetically Modified Mosquitoes (November 8 lecture)
        \end{itemize}
        \item Problems:
        \begin{itemize}
            \item Mosquitoes are evolving evolution to BOTH bed nets and DDT
        \end{itemize}
        \item Poverty $\iff$ Malaria in a lot of places
        \begin{itemize}
            \item Farmers get sick right when their crops are ready
        \end{itemize}
    \end{itemize}

\end{document}
















