\documentclass{article}


\usepackage[margin=0.6in]{geometry}
\usepackage{amssymb, amsmath, amsfonts}
\usepackage{mathtools}
\usepackage{physics}
\usepackage{enumerate}
\usepackage{array}
\newcommand{\Rl}{\mathbb{R}}
\newcommand{\f}[3]{#1\ :\ #2 \rightarrow #3}

\title{PMI 214 Notes - ``West Nile virus: Epidemiology and Ecology''}
\author{Sam Fleischer - Speaker: Aaron C.~Brault (Centers for Disease Control and Prevention - Fort Collins, CO)}
\date{October 4, 2016}

\begin{document}
    \maketitle

    \section{Key Concepts}
        \begin{itemize}
            \item Basic epi (transmission cycles and surveillance)
            \item ...
        \end{itemize}

    \section{types}
        \begin{itemize}
            \item Febrile w/ Arthalgia
            \begin{itemize}
                \item WNV
                \item DEN
                \item 
            \end{itemize}
            \item ...
        \end{itemize}

    \begin{itemize}
        \item CA as multiple flaviviruses interacting
        \item Enzootic (maintenance/amplification)
        \begin{itemize}
            \item mosquitoes feed on birds, which perpetuate the cycle
        \end{itemize}
        \item epidemic if mosquito bites a person
        \begin{itemize}
            \item but people don't necessarily contribute to the cycle.. they are dead end hosts but can become symptomatic
        \end{itemize}
        \item epizootic
        \begin{itemize}
            \item dead crows (early sentinel warning system.. people notice when a bunch of birds drop out of the sky) as well as horses
        \end{itemize}
        \item lots of genetic diversity among WNV
        \item only 1/5 (VERY approximate) cases of human infections are symptomatic
        \begin{itemize}
            \item acute systemic febrile illness
            \item headache, myalgia, arthralgia, rash, GI symptoms
        \end{itemize}
        \item <1\% neuroinvasive disease
        \begin{itemize}
            \item encephalitis, meningitis, acute flaccid paralysis (polio-like syndrome)
            \item incidence highest among persons $\geq 60$ years
            \item case fatality $\approx 10\%$
        \end{itemize}
        \item west nile activity is focused in cities or counties
        \item number of WNV in blood donors matches overall data
        \item transmission picks up in summer (probably in july/august) but humans report in august/september
        \item older populations are more susceptible to WNV
        \item risk factors:
        \begin{itemize}
            \item being male because
            \begin{itemize}
                \item more exposure risk because of hobbies/vocation?
                \item men have underlying health conditions which make them more susceptible to contracting the disease
            \end{itemize}
            \item diabetes, hypertension, and other co-morbidities
            \item immune suppression
        \end{itemize}
        \item virus moved east coast to west from 1999 to 2005
        \item areas with wet springs and above-average summer temps are prone to WNV transmission.. ex:
        \begin{itemize}
            \item dallas in 2012
            \item chiacgo in 2002
        \end{itemize}
        \item virus moves through mosquito faster in hotter temps
        \item three phases of WNV
        \begin{itemize}
            \item indentified infected mosquitoes or identify dead birds
            \item next year, big blow up in human cases, sentinel system goes crazy.. lots of dead birds
            \item third year, subsidence period - maybe high rates in birds, fewer infected mosquitoes (probably due to herd immunity in birds), lower rates in humans
        \end{itemize}
        \item american crows highly sensitive to infection (nearly 100\% lethal if infected)
        \begin{itemize}
            \item very high viral titres
        \end{itemize}
        \item other birds
        \begin{itemize}
            \item american crows
            \item blue jays
            \item western scrub jays (16\%)
            \item yellow billed magpies
            \item house sparrow
        \end{itemize}
        \item Davis in particular is a hybrid zone for both north and south \# 1 mosquito for WNV
    \end{itemize}
    \section{WNV Infection in horses}
        \begin{itemize}
            \item big spread in 2002 - followed same basic outbreak pattern as in humans
            \item horses are dead-end hosts (like humans)
            \item viremia - brief and low magnitude
            \item 1/20 horses develop viremia
            \item fatality is 25-40\% of sick hroses
            \item WNV therapy for horses.. not too effective
            \begin{itemize}
                \item best is mosquito control and vaccinations
            \end{itemize}
            \item decrease of WNV in horses in 2003 presumably largely due to vaccine effort
            \item viremia is proportional to likelihood of succombing to infection
            \item cost of being a non-vector-borne disease is higher than being a vector-borne disease.  This is because people sick with influenze stay home
            \item but benefit of a vector-borne disease is lower than benefit of a non-vector-borne disease
            \item both these slides show that vector-borne diseases evolve higher virulence than non-vector-borne diseases
            \item slow but steady upward trend of higher viremia in birds, which implies higher virulence (response to herd-immunity?)
            \item birds adapting as well - birds steadily evolving resistant phenotype
            \item genetics ARMS RACE keeps the number of virus about steady
        \end{itemize}

        contact
        \begin{itemize}
            \item abrault@cdc.gov
            \item 970-266-3517
        \end{itemize}

\end{document}
















