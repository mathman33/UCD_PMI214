\documentclass{article}


\usepackage[margin=0.6in]{geometry}
\usepackage{amssymb, amsmath, amsfonts}
\usepackage{mathtools}
\usepackage{physics}
\usepackage{enumerate}
\usepackage{array}
\newcommand{\Rl}{\mathbb{R}}
\newcommand{\f}[3]{#1\ :\ #2 \rightarrow #3}

\title{PMI 214 Notes}
\author{Sam Fleischer}
\date{October 25, 2016}

\begin{document}
    \maketitle

    \section{Entomological Inoculation Rate}
        \begin{itemize}
            \item Rate at which a person is bitten by infectious vectors per day
            \begin{align}
                EIR = ma\phi\theta
            \end{align}
            where
            \begin{itemize}
                \item $m$ is the number of mosquitoes
                \item $a$ is the number of bites per day
                \item $\phi$ is the faction of bites which are on humans
                \item $\theta$ is the fraction of vectors which are transmitting the disease
            \end{itemize}
        \end{itemize}
    \section{What is ``mathematical epidemilogy?''}
        \begin{itemize}
            \item based on theory of transmission process
            \item provides a conceptual representation of what is known about the mechanisms of a system
            \item helps identify knowledge gaps
            \item often used for hypothesis testing (tests scenarios that cannot be easily/safely tested in nature)
        \end{itemize}

    \section{Vectorial Capacity}
        \begin{align}
            C = \frac{ma^2b}{-\ln p}p^n
        \end{align}
        $C = $ number of infective vector bites that would arise from all the mosquitoes that bite a single host on a single day
        \begin{itemize}
            \item $m$, $a$ same as before
            \item $b$ is the vector competence
            \item $n$ is the extrinsic incubation period for pathogen
            \item $p$ is the daily mosquito survival probability
        \end{itemize}

    \section{Basic Reproductive Rate}
        \begin{align}
            R_0 = \frac{C}{r}
        \end{align}
        $R_0 = $ the avg.~number of future host infections that will arise following introduction of a single infectious host in a susceptible population
        \begin{itemize}
            \item $r$ is the host recovery rate... unit: $1/(\text{infectious period})$
        \end{itemize}

    \section{SIR/SEIR}
        For hosts, susceptible $\rightarrow$ exposed $\rightarrow$ infective $\rightarrow$ removed.  For vectors, no exposed period.

\end{document}
















