\documentclass{article}


\usepackage[margin=0.6in]{geometry}
\usepackage{amssymb, amsmath, amsfonts}
\usepackage{mathtools}
\usepackage{physics}
\usepackage{enumerate}
\usepackage{array}
\newcommand{\Rl}{\mathbb{R}}
\newcommand{\f}[3]{#1\ :\ #2 \rightarrow #3}

\title{PMI 214 Notes}
\author{Sam Fleischer - Speaker: Eva Harris\\ (Professor, Division of Infectious Diseases and Vaccinology \\ Director, Center for Global Public Health \\ School of Public Health, UC Berkeley)}
\date{October 13, 2016}

\begin{document}
    \maketitle

    \begin{itemize}
        \item 4 types of Dengue
        \item If you have antibodies to DENV-1 you are MORE susceptible to DENV-2, 3, and 4.  In other diseases, antibodies help establish immunity
        \item Frequent spikes in Dengue (every 4 years) growing in magnitude
        \item WWII troop movement caused initial spread - now its global trade, travel, urbanization
        \item Dengue vectors (aedes) have perfectly adapted to human ecology in the 21st century
    \end{itemize}

\end{document}
















