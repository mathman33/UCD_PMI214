\documentclass{article}


\usepackage[margin=0.6in]{geometry}
\usepackage{amssymb, amsmath, amsfonts}
\usepackage{mathtools}
\usepackage{physics}
\usepackage{enumerate}
\usepackage{array}
\newcommand{\Rl}{\mathbb{R}}
\newcommand{\f}[3]{#1\ :\ #2 \rightarrow #3}

\title{PMI 214 Notes}
\author{Sam Fleischer}
\date{November 3, 2016}

\begin{document}
    \maketitle

    \begin{itemize}
        \item WHO definition of ``Integrated Vector Management'' is defined as a ``rational decision-making process for the optimal use of resources for vector control''
        \item Five key elements:
        \begin{itemize}
            \item Integrated approaches
            \begin{itemize}
                \item The desired outcome is to reduce disease.  Controlling the vector is a means to the ultimate goal.  Almost all IVM approaches include uses of drugs and vector control methods
                \item IVM strategies are supposed to \emph{minimize} the use of chemicals
            \end{itemize}
            \item Capaciy building
            \begin{itemize}
                \item Must develop physical infrastructure, control some financial resources
            \end{itemize}
            \item collaboration within the health sector and other sectors
            \item Advocacy, social mobilization, and legislation
            \begin{itemize}
                \item Vector control is not in a vacuum
                \item Regulatory and legislative controls are important - must have relationships with politicians
                \item Empowerment of communities - currently not well implemented, but arguably the most important means of vector control - we must educate and then rely on the public
            \end{itemize}
            \item evidence-based decision-making
            \begin{itemize}
                \item Using solid academic research to make decisions
            \end{itemize}
        \end{itemize}
        \item Key challenges to successful vector control is the lack of intersectional collaboration
        \begin{itemize}
            \item govt. ministries
            \item municipal entities
            \item stake-holders in communities
        \end{itemize}
        \item Adulticide
        \begin{itemize}
            \item Paris Green (Toxic)
            \item DDT, Lindane (Environmental concerns, and resistance)
            \item Organophosphates carbamates (Environmental concerns, and resistance)
            \item Pyrethroids (Larviciding, environmental concerns, and resistance)
        \end{itemize}
        \item Larvicide
        \begin{itemize}
            \item Petroleum Oil (Toxic)
            \item Vegetable oil
            \item fish
            \item B.t.i. and B.s. (resistance?)
            \item Methroprene (IGR) (resistance?)
        \end{itemize}
        \item An IVM strategy should include as much source reduction or control measures as possilbe
        \begin{itemize}
            \item In Africa, pools on mud roads
            \item In California
            \begin{itemize}
                \item underground rain drains
                \item PG\&E electricity vaults
            \end{itemize}
        \end{itemize}
        \item Mosquitoes are exhibiting behavioral resistance to avoid chemical exposure
        \begin{itemize}
            \item Widespread use of bed-nets caused massive selection for the phenotype allowing mosquitoes to feed at sunset, rather than at night.
        \end{itemize}
        \item There are secondary vectors for Malaria - some nonzero percentage of transmission.
        \item Incompatibility Insect Technique - releasing sterile males into the population.
        \item Traps
        \item Parasitic fungi on mosquitoes
        \item Rice cultivation is creating major breeding sources of malaria vectors
        \item Canals are a problem - must have deep, fast-flowing canals - not slow-flowing shallow canals
    \end{itemize}

\end{document}
















